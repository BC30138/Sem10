\documentclass{article}
\usepackage[utf8]{inputenc}
\usepackage[english,russian]{babel}
\usepackage{amsmath}
\usepackage{enumerate}
\usepackage[12pt]{extsizes}
\usepackage{xcolor,listings}
\usepackage[left=30mm, top=20mm, right=20mm, bottom=20mm, nohead, footskip=10mm]{geometry}


\usepackage[absolute,overlay]{textpos}
\usepackage{indentfirst}
\usepackage{float}
\restylefloat{table}
\usepackage{hyperref}
\usepackage{mathtext}
\usepackage{amsfonts}
\usepackage{amsthm}
\usepackage{tikz}
\usetikzlibrary{shapes,positioning,shadows,trees,automata,arrows.meta,shapes.geometric}
\usepackage{pgf-pie}
\usepackage{chngcntr}
\usepackage{pdfpages}
\usepackage{systeme}
\usepackage{empheq}
\usepackage{subcaption}
\counterwithin{figure}{section}

\pagestyle{plain}

\definecolor{String}{RGB}{134, 179, 0}
\definecolor{KeyColor}{RGB}{160,0,102}

\captionsetup[lstlisting]{
	singlelinecheck=false,
	margin=3pt,
	font=,skip=5pt,
	font={bf}
}

\lstdefinestyle{style}{
	aboveskip=0pt,
	belowskip=10pt,
	showspaces=false,
	showstringspaces=false,
	basicstyle=\ttfamily\footnotesize,
	numbers=left,
	% texcl=true,
	breaklines=true,
	postbreak=\mbox{\textcolor{red}{$\hookrightarrow$}\space},
	keywordstyle=\color{KeyColor},
	identifierstyle=\color{black},
	numberstyle=\scriptsize,
	stringstyle=\color{String},
	commentstyle=\color{gray},
	frame=tb
}

\begin{document}
    \thispagestyle{empty}
	\begin{center}
		Санкт-Петербургский политехнический университет Петра Великого\\
		Институт прикладной математики и механики\\
		Кафедра <<Телематика (при ЦНИИ РТК)>>\\
		\vspace*{\fill}
		\textbf{\Large{КУРСОВАЯ РАБОТА}}\\
		\vspace{0.5cm}
        \large{по дисциплине <<Семинар по роботизированным системам>>\\}
        \large{на тему <<Муравьиный алгоритм и алгоритм коллективного распределения целей>>}\\
        \vspace{1cm}
        по направлению 02.04.01.02 <<Организация и управление суперкомпьютерными системами>>
	\end{center}
	\vspace{3cm}
	\begin{tabular} {l l l}
	\hspace{10cm} & Выполнил: & Титов А.И.\\
	& Проверил: & Глазунов В.В.
	\end{tabular}
	\vspace*{\fill}
	\begin{center}
		Санкт-Петербург\\
		2019
    \end{center}
    \newpage


	\renewcommand\contentsname{Оглавление}
	\tableofcontents

	\newpage
	\addcontentsline{toc}{section}{ПОСТАНОВКА ЗАДАЧИ}
	\section*{ПОСТАНОВКА ЗАДАЧИ}

	Целью курсовой работы является реализация и исследование алгоритмов для построения оптимальных путей роботов до целей. Далее под роботом, для простоты, будет иметься в виду непосредственно начальная координата пути, а под целью, соответственно, конечная координата.

	Таким образом, требуется создать карту местности, на которой помещается набор роботов и набор целей, после чего каждому роботу оптимально назначить цель и построить путь до нее.

	Для этого требуется реализовать следующие алгоритмы:

    \begin{itemize}
        \item Алгоритм для процедурного построения реалистичной карты местности \cite{terrain};
        \item Алгоритм коллективного распределения целей \cite{plan};
        \item Алгоритм поиска пути \cite{path}.
	\end{itemize}

	Для генерации реалистичной карты среды выбран алгоритм Diamond-Square \cite{DS}. Алгоритм коллективного распределения целей описан в \cite{plan} в главе <<Алгоритм коллективного улучшения плана 3.7>> на стр. 102. Для вычисления пути от робота до цели рассматривается муравьиный алгоритм \cite{ant}.

	Выполняются следующие задачи для достижения цели:
	\begin{enumerate}
		\item Реализация алгоритма Diamond-Square;
		\item Реализация муравьиного алгоритма;
		\item Реализация алгоритма коллективного распределения целей;
		\item Исследования реализованного функционала.
	\end{enumerate}

	Реализация осуществляется на языке Python. Исследование реализованного функционала заключается в следующем:

	\begin{enumerate}
		\item Генерация 10 различных карт для каждого размера из: 25x25, 50x50, 100x100, 250x250, 500x500, 1000x1000;
		\item На каждой из карт сгенерированных генерируются наборы роботов и, соответственно, цели к ним в численности: 5, 10, 20, 50 (наборов каждого размера для каждой карты тоже должно быть по 10, но из соображений производительности этот пункт опущен);
		\item Для заданных наборов распределяются цели по роботам;
		\item Для каждого построенного набора строятся графики, отображающие зависимость времени выполнения программы от размеров карт и численности роботов. (В изначальном задании указано построить график содержащий все измеренные времена, но для наглядности графики строятся только средних элементов замеров, а полную картину отражают таблицы в \hyperref[sec:time]{ПРИЛОЖЕНИЕ Б});
		\item Теоретическое исследование реализуемого функционала.
	\end{enumerate}

	\newpage
    \section{Описание алгоритмов}

    \subsection{Алгоритм Diamond-Square}

    \subsection{Муравьиный алгоритм}

    \subsection{Алгоритм коллективного распределения целей}

    \newpage
    \section{Программная реализация}

	\newpage
    \section{Результаты}

    \newpage
    \addcontentsline{toc}{section}{ЗАКЛЮЧЕНИЕ}
    \section*{ЗАКЛЮЧЕНИЕ}

    \newpage
	\renewcommand\refname{ЛИТЕРАТУРА}
	\addcontentsline{toc}{section}{ЛИТЕРАТУРА}
	\begin{thebibliography}{}
		\bibitem{terrain} Miguel Monteiro de Sousa Frade. Genetic Terrain Programming // Universidad de Extremadura, 2008, pp. 103
		\bibitem{plan} Каляев И.А. Модели и алгоритмы коллективного управления в группах роботов // Физматлит, 2009, 279с.
		\bibitem{path} Gregor Klančar. Path Planning // Wheeled Mobile Robotics, 2017, pp. 161-206
		\bibitem{DS} Jacob Olsen. Realtime Procedural Terrain Generation // University of Southern Denmark, 2004, pp. 20
		\bibitem{ant} M. Brand, M. Masuda, N. Wehner, X.-H. Yu. Ant colony optimization algorithm for robot path planning // Computer Design and Applications (ICCDA) 2010 International Conference on, vol. 3, 2010, pp. 436-440.
	\end{thebibliography}


    \newpage
    \addcontentsline{toc}{section}{ПРИЛОЖЕНИЕ А. Исходный код}
	\section*{ПРИЛОЖЕНИЕ А. Исходный код} \label{sec:code}
	Ниже приведен исходный код на языке Python
	\lstinputlisting[language = python, style=style, title={main.py}]{data/code/main.py}
	\lstinputlisting[language = python, style=style, title={graph.py}]{data/code/graph.py}
	\lstinputlisting[language = python, style=style, title={ant.py}]{data/code/ant.py}
	\lstinputlisting[language = python, style=style, title={planning.py}]{data/code/planning.py}
	\lstinputlisting[language = python, style=style, title={tools.py}]{data/code/tools.py}


    \newpage
    \addcontentsline{toc}{section}{ПРИЛОЖЕНИЕ Б. Таблицы замеров времени}
	\section*{ПРИЛОЖЕНИЕ Б. Таблицы замеров времени} \label{sec:time}
	Ниже приведены замеры времени (c.) муравьиного алгоритма для каждой сгенерированной карты и каждого количества роботов:

	\begin{table}[H]
\centering
\begin{tabular}{|r|l|l|l|l|}
\hline
№ карты\textbackslash Кол-во роботов & \textbf{5} & \textbf{10} & \textbf{20} & \textbf{50}\\ \hline
1 & 1.5568 & 1.69588 & 2.75487 & 4.2076\\ \hline
2 & 1.69226 & 1.80428 & 3.21331 & 4.2441\\ \hline
3 & 1.09952 & 1.34256 & 2.66641 & 6.43867\\ \hline
4 & 0.96775 & 3.23578 & 2.67067 & 4.53471\\ \hline
5 & 0.80875 & 1.89084 & 3.0447 & 5.53788\\ \hline
6 & 1.1738 & 2.62321 & 2.6756 & 5.01334\\ \hline
7 & 1.40806 & 1.95556 & 2.0411 & 6.20485\\ \hline
8 & 1.08748 & 2.11052 & 2.38643 & 5.60215\\ \hline
9 & 1.26882 & 1.84033 & 2.1164 & 5.56547\\ \hline
10 & 0.73943 & 1.60901 & 3.37735 & 5.15904\\ \hline
Средний элемент & 1.09952 & 1.84033 & 2.67067 & 5.15904\\ \hline
\end{tabular}
\caption*{Размер карты: 25x25}
\end{table}

	\begin{table}[H]
\centering
\begin{tabular}{|r|l|l|l|l|}
\hline
№ карты\textbackslash Кол-во роботов & \textbf{5} & \textbf{10} & \textbf{20} & \textbf{50}\\ \hline
1 & 2.15365 & 3.59408 & 6.54677 & 10.73975\\ \hline
2 & 1.71077 & 3.3017 & 6.41902 & 11.11771\\ \hline
3 & 3.33052 & 2.96686 & 7.60223 & 12.24219\\ \hline
4 & 2.48163 & 4.87254 & 9.19686 & 12.41132\\ \hline
5 & 3.9287 & 3.95599 & 6.41092 & 11.49145\\ \hline
6 & 1.6465 & 4.43041 & 7.68628 & 13.13047\\ \hline
7 & 2.31836 & 4.70432 & 5.39536 & 13.06448\\ \hline
8 & 1.93215 & 2.44238 & 5.48949 & 12.79632\\ \hline
9 & 2.99847 & 2.71091 & 7.70761 & 15.13225\\ \hline
10 & 2.68102 & 4.48696 & 6.36677 & 15.12783\\ \hline
Средний элемент & 2.31836 & 3.59408 & 6.41902 & 12.41132\\ \hline
\end{tabular}
\caption*{Размер карты: 50x50}
\end{table}

	\begin{table}[H]
\centering
\begin{tabular}{|r|l|l|l|l|}
\hline
№ карты\textbackslash Кол-во роботов & \textbf{5} & \textbf{10} & \textbf{20} & \textbf{50}\\ \hline
1 & 6.32895 & 8.73452 & 14.94095 & 24.71504\\ \hline
2 & 5.26532 & 9.12678 & 13.68367 & 30.10719\\ \hline
3 & 4.87571 & 12.12137 & 14.66411 & 26.09427\\ \hline
4 & 5.98595 & 7.40209 & 12.80597 & 23.22383\\ \hline
5 & 3.94488 & 10.1101 & 17.56255 & 28.4082\\ \hline
6 & 7.12389 & 8.61739 & 13.43984 & 25.22285\\ \hline
7 & 5.36162 & 8.51896 & 13.89583 & 25.27979\\ \hline
8 & 5.02578 & 9.56457 & 11.74729 & 23.15173\\ \hline
9 & 5.34449 & 13.91983 & 14.71402 & 22.08825\\ \hline
10 & 6.81967 & 9.45882 & 17.0239 & 31.06226\\ \hline
Средний элемент & 5.34449 & 9.12678 & 13.89583 & 25.22285\\ \hline
\end{tabular}
\caption*{Размер карты: 100x100}
\end{table}

	\begin{table}[H]
\centering
\begin{tabular}{|r|l|l|l|l|}
\hline
№ карты\textbackslash Кол-во роботов & \textbf{5} & \textbf{10} & \textbf{20} & \textbf{50}\\ \hline
1 & 42.28 & 32.36 & 52.149 & 107.196\\ \hline
2 & 19.016 & 27.002 & 69.753 & 103.928\\ \hline
3 & 20.361 & 30.919 & 45.16 & 103.125\\ \hline
4 & 17.649 & 38.522 & 52.348 & 115.323\\ \hline
5 & 19.154 & 39.692 & 60.4 & 123.056\\ \hline
6 & 25.561 & 25.462 & 49.701 & 121.093\\ \hline
7 & 18.231 & 34.012 & 58.345 & 93.992\\ \hline
8 & 13.386 & 34.576 & 51.704 & 105.007\\ \hline
9 & 17.922 & 35.839 & 48.99 & 98.7\\ \hline
10 & 23.871 & 26.686 & 54.966 & 103.896\\ \hline
Средний элемент & 19.016 & 32.36 & 52.149 & 103.928\\ \hline
\end{tabular}
\caption*{Размер карты: 250x250}
\end{table}

	\begin{table}[H]
\centering
\begin{tabular}{|r|l|l|l|l|}
\hline
№ карты\textbackslash Кол-во роботов & \textbf{5} & \textbf{10} & \textbf{20} & \textbf{50}\\ \hline
1 & 0.0 & 0.0 & 0.0 & 0.003\\ \hline
2 & 0.0 & 0.0 & 0.0 & 0.004\\ \hline
3 & 0.0 & 0.0 & 0.0 & 0.004\\ \hline
4 & 0.0 & 0.0 & 0.001 & 0.003\\ \hline
5 & 0.0 & 0.0 & 0.001 & 0.003\\ \hline
6 & 0.0 & 0.0 & 0.0 & 0.003\\ \hline
7 & 0.0 & 0.0 & 0.0 & 0.004\\ \hline
8 & 0.0 & 0.0 & 0.0 & 0.003\\ \hline
9 & 0.0 & 0.0 & 0.001 & 0.003\\ \hline
10 & 0.0 & 0.0 & 0.0 & 0.003\\ \hline
Средний элемент & 0.0 & 0.0 & 0.0 & 0.003\\ \hline
\end{tabular}
\caption*{Размер карты: 500x500}
\end{table}

	\begin{table}[H]
\centering
\begin{tabular}{|r|l|l|l|l|}
\hline
№ карты\textbackslash Кол-во роботов & \textbf{5} & \textbf{10} & \textbf{20} & \textbf{50}\\ \hline
1 & 985.467 & 292.7 & 847.499 & 2510.479\\ \hline
2 & 289.546 & 551.286 & 879.198 & 1478.341\\ \hline
3 & 775.556 & 516.55 & 493.118 & 2206.797\\ \hline
4 & 612.462 & 1196.722 & 596.369 & 2167.358\\ \hline
5 & 356.076 & 626.026 & 757.651 & 2049.006\\ \hline
6 & 258.133 & 1190.344 & 1356.154 & 2417.786\\ \hline
7 & 1100.106 & 554.877 & 1412.945 & 2335.377\\ \hline
8 & 715.701 & 372.302 & 1571.88 & 4075.243\\ \hline
9 & 905.145 & 777.733 & 2050.21 & 4158.513\\ \hline
10 & 170.516 & 1581.04 & 2182.638 & 4512.163\\ \hline
Средний элемент & 612.462 & 554.877 & 879.198 & 2335.377\\ \hline
\end{tabular}
\caption*{Размер карты: 1000x1000}
\end{table}


	Ниже приведены замеры времени (c.) алгоритма планирования для каждой сгенерированной карты и каждого количества роботов:

	\begin{table}[H]
\centering
\begin{tabular}{|r|l|l|l|l|}
\hline
№ карты\textbackslash Кол-во роботов & \textbf{5} & \textbf{10} & \textbf{20} & \textbf{50}\\ \hline
1 & 1.5568 & 1.69588 & 2.75487 & 4.2076\\ \hline
2 & 1.69226 & 1.80428 & 3.21331 & 4.2441\\ \hline
3 & 1.09952 & 1.34256 & 2.66641 & 6.43867\\ \hline
4 & 0.96775 & 3.23578 & 2.67067 & 4.53471\\ \hline
5 & 0.80875 & 1.89084 & 3.0447 & 5.53788\\ \hline
6 & 1.1738 & 2.62321 & 2.6756 & 5.01334\\ \hline
7 & 1.40806 & 1.95556 & 2.0411 & 6.20485\\ \hline
8 & 1.08748 & 2.11052 & 2.38643 & 5.60215\\ \hline
9 & 1.26882 & 1.84033 & 2.1164 & 5.56547\\ \hline
10 & 0.73943 & 1.60901 & 3.37735 & 5.15904\\ \hline
Средний элемент & 1.09952 & 1.84033 & 2.67067 & 5.15904\\ \hline
\end{tabular}
\caption*{Размер карты: 25x25}
\end{table}

	\begin{table}[H]
\centering
\begin{tabular}{|r|l|l|l|l|}
\hline
№ карты\textbackslash Кол-во роботов & \textbf{5} & \textbf{10} & \textbf{20} & \textbf{50}\\ \hline
1 & 2.15365 & 3.59408 & 6.54677 & 10.73975\\ \hline
2 & 1.71077 & 3.3017 & 6.41902 & 11.11771\\ \hline
3 & 3.33052 & 2.96686 & 7.60223 & 12.24219\\ \hline
4 & 2.48163 & 4.87254 & 9.19686 & 12.41132\\ \hline
5 & 3.9287 & 3.95599 & 6.41092 & 11.49145\\ \hline
6 & 1.6465 & 4.43041 & 7.68628 & 13.13047\\ \hline
7 & 2.31836 & 4.70432 & 5.39536 & 13.06448\\ \hline
8 & 1.93215 & 2.44238 & 5.48949 & 12.79632\\ \hline
9 & 2.99847 & 2.71091 & 7.70761 & 15.13225\\ \hline
10 & 2.68102 & 4.48696 & 6.36677 & 15.12783\\ \hline
Средний элемент & 2.31836 & 3.59408 & 6.41902 & 12.41132\\ \hline
\end{tabular}
\caption*{Размер карты: 50x50}
\end{table}

	\begin{table}[H]
\centering
\begin{tabular}{|r|l|l|l|l|}
\hline
№ карты\textbackslash Кол-во роботов & \textbf{5} & \textbf{10} & \textbf{20} & \textbf{50}\\ \hline
1 & 6.32895 & 8.73452 & 14.94095 & 24.71504\\ \hline
2 & 5.26532 & 9.12678 & 13.68367 & 30.10719\\ \hline
3 & 4.87571 & 12.12137 & 14.66411 & 26.09427\\ \hline
4 & 5.98595 & 7.40209 & 12.80597 & 23.22383\\ \hline
5 & 3.94488 & 10.1101 & 17.56255 & 28.4082\\ \hline
6 & 7.12389 & 8.61739 & 13.43984 & 25.22285\\ \hline
7 & 5.36162 & 8.51896 & 13.89583 & 25.27979\\ \hline
8 & 5.02578 & 9.56457 & 11.74729 & 23.15173\\ \hline
9 & 5.34449 & 13.91983 & 14.71402 & 22.08825\\ \hline
10 & 6.81967 & 9.45882 & 17.0239 & 31.06226\\ \hline
Средний элемент & 5.34449 & 9.12678 & 13.89583 & 25.22285\\ \hline
\end{tabular}
\caption*{Размер карты: 100x100}
\end{table}

	\begin{table}[H]
\centering
\begin{tabular}{|r|l|l|l|l|}
\hline
№ карты\textbackslash Кол-во роботов & \textbf{5} & \textbf{10} & \textbf{20} & \textbf{50}\\ \hline
1 & 42.28 & 32.36 & 52.149 & 107.196\\ \hline
2 & 19.016 & 27.002 & 69.753 & 103.928\\ \hline
3 & 20.361 & 30.919 & 45.16 & 103.125\\ \hline
4 & 17.649 & 38.522 & 52.348 & 115.323\\ \hline
5 & 19.154 & 39.692 & 60.4 & 123.056\\ \hline
6 & 25.561 & 25.462 & 49.701 & 121.093\\ \hline
7 & 18.231 & 34.012 & 58.345 & 93.992\\ \hline
8 & 13.386 & 34.576 & 51.704 & 105.007\\ \hline
9 & 17.922 & 35.839 & 48.99 & 98.7\\ \hline
10 & 23.871 & 26.686 & 54.966 & 103.896\\ \hline
Средний элемент & 19.016 & 32.36 & 52.149 & 103.928\\ \hline
\end{tabular}
\caption*{Размер карты: 250x250}
\end{table}

	\begin{table}[H]
\centering
\begin{tabular}{|r|l|l|l|l|}
\hline
№ карты\textbackslash Кол-во роботов & \textbf{5} & \textbf{10} & \textbf{20} & \textbf{50}\\ \hline
1 & 0.0 & 0.0 & 0.0 & 0.003\\ \hline
2 & 0.0 & 0.0 & 0.0 & 0.004\\ \hline
3 & 0.0 & 0.0 & 0.0 & 0.004\\ \hline
4 & 0.0 & 0.0 & 0.001 & 0.003\\ \hline
5 & 0.0 & 0.0 & 0.001 & 0.003\\ \hline
6 & 0.0 & 0.0 & 0.0 & 0.003\\ \hline
7 & 0.0 & 0.0 & 0.0 & 0.004\\ \hline
8 & 0.0 & 0.0 & 0.0 & 0.003\\ \hline
9 & 0.0 & 0.0 & 0.001 & 0.003\\ \hline
10 & 0.0 & 0.0 & 0.0 & 0.003\\ \hline
Средний элемент & 0.0 & 0.0 & 0.0 & 0.003\\ \hline
\end{tabular}
\caption*{Размер карты: 500x500}
\end{table}

	\begin{table}[H]
\centering
\begin{tabular}{|r|l|l|l|l|}
\hline
№ карты\textbackslash Кол-во роботов & \textbf{5} & \textbf{10} & \textbf{20} & \textbf{50}\\ \hline
1 & 985.467 & 292.7 & 847.499 & 2510.479\\ \hline
2 & 289.546 & 551.286 & 879.198 & 1478.341\\ \hline
3 & 775.556 & 516.55 & 493.118 & 2206.797\\ \hline
4 & 612.462 & 1196.722 & 596.369 & 2167.358\\ \hline
5 & 356.076 & 626.026 & 757.651 & 2049.006\\ \hline
6 & 258.133 & 1190.344 & 1356.154 & 2417.786\\ \hline
7 & 1100.106 & 554.877 & 1412.945 & 2335.377\\ \hline
8 & 715.701 & 372.302 & 1571.88 & 4075.243\\ \hline
9 & 905.145 & 777.733 & 2050.21 & 4158.513\\ \hline
10 & 170.516 & 1581.04 & 2182.638 & 4512.163\\ \hline
Средний элемент & 612.462 & 554.877 & 879.198 & 2335.377\\ \hline
\end{tabular}
\caption*{Размер карты: 1000x1000}
\end{table}


    \newpage
    \addcontentsline{toc}{section}{ПРИЛОЖЕНИЕ В. Графики решений}
	\section*{ПРИЛОЖЕНИЕ В. Графики решений} \label{sec:png}
	Ниже представлены построенные пути с средним временем выполнения для разного числа роботов и разных размеров матриц (черным отмечены роботы, красным - цели):
	\begin{table}[H]
		\begin{tabular}{c c}
			\begin{subfigure}{0.5\linewidth}
				\includegraphics[width = 1.0\columnwidth]{data/mean_paths/25x25/5.png}
			\caption*{5 роботов}
			\end{subfigure}
			&
			\begin{subfigure}{0.5\linewidth}
				\includegraphics[width = 1.0\columnwidth]{data/mean_paths/25x25/10.png}
			\caption*{10 роботов}
			\end{subfigure}
			\\
            \begin{subfigure}{0.5\linewidth}
				\includegraphics[width = 1.0\columnwidth]{data/mean_paths/25x25/20.png}
			\caption*{20 роботов}
			\end{subfigure}
			&
			\begin{subfigure}{0.5\linewidth}
				\includegraphics[width = 1.0\columnwidth]{data/mean_paths/25x25/50.png}
			\caption*{50 роботов}
			\end{subfigure}
        \end{tabular}
        \caption*{Размер карты: 25x25}
	\end{table}

	\begin{table}[H]
		\begin{tabular}{c c}
			\begin{subfigure}{0.5\linewidth}
				\includegraphics[width = 1.0\columnwidth]{data/mean_paths/50x50/5.png}
			\caption*{5 роботов}
			\end{subfigure}
			&
			\begin{subfigure}{0.5\linewidth}
				\includegraphics[width = 1.0\columnwidth]{data/mean_paths/50x50/10.png}
			\caption*{10 роботов}
			\end{subfigure}
			\\
            \begin{subfigure}{0.5\linewidth}
				\includegraphics[width = 1.0\columnwidth]{data/mean_paths/50x50/20.png}
			\caption*{20 роботов}
			\end{subfigure}
			&
			\begin{subfigure}{0.5\linewidth}
				\includegraphics[width = 1.0\columnwidth]{data/mean_paths/50x50/50.png}
			\caption*{50 роботов}
			\end{subfigure}
        \end{tabular}
        \caption*{Размер карты: 50x50}
	\end{table}

	\begin{table}[H]
		\begin{tabular}{c c}
			\begin{subfigure}{0.5\linewidth}
				\includegraphics[width = 1.0\columnwidth]{data/mean_paths/100x100/5.png}
			\caption*{5 роботов}
			\end{subfigure}
			&
			\begin{subfigure}{0.5\linewidth}
				\includegraphics[width = 1.0\columnwidth]{data/mean_paths/100x100/10.png}
			\caption*{10 роботов}
			\end{subfigure}
			\\
            \begin{subfigure}{0.5\linewidth}
				\includegraphics[width = 1.0\columnwidth]{data/mean_paths/100x100/20.png}
			\caption*{20 роботов}
			\end{subfigure}
			&
			\begin{subfigure}{0.5\linewidth}
				\includegraphics[width = 1.0\columnwidth]{data/mean_paths/100x100/50.png}
			\caption*{50 роботов}
			\end{subfigure}
        \end{tabular}
        \caption*{Размер карты: 100x100}
	\end{table}

	\begin{table}[H]
		\begin{tabular}{c c}
			\begin{subfigure}{0.5\linewidth}
				\includegraphics[width = 1.0\columnwidth]{data/mean_paths/250x250/5.png}
			\caption*{5 роботов}
			\end{subfigure}
			&
			\begin{subfigure}{0.5\linewidth}
				\includegraphics[width = 1.0\columnwidth]{data/mean_paths/250x250/10.png}
			\caption*{10 роботов}
			\end{subfigure}
			\\
            \begin{subfigure}{0.5\linewidth}
				\includegraphics[width = 1.0\columnwidth]{data/mean_paths/250x250/20.png}
			\caption*{20 роботов}
			\end{subfigure}
			&
			\begin{subfigure}{0.5\linewidth}
				\includegraphics[width = 1.0\columnwidth]{data/mean_paths/250x250/50.png}
			\caption*{50 роботов}
			\end{subfigure}
        \end{tabular}
        \caption*{Размер карты: 250x250}
	\end{table}

	\begin{table}[H]
		\begin{tabular}{c c}
			\begin{subfigure}{0.5\linewidth}
				\includegraphics[width = 1.0\columnwidth]{data/mean_paths/1000x1000/5.png}
			\caption*{5 роботов}
			\end{subfigure}
			&
			\begin{subfigure}{0.5\linewidth}
				\includegraphics[width = 1.0\columnwidth]{data/mean_paths/1000x1000/10.png}
			\caption*{10 роботов}
			\end{subfigure}
			\\
            \begin{subfigure}{0.5\linewidth}
				\includegraphics[width = 1.0\columnwidth]{data/mean_paths/1000x1000/20.png}
			\caption*{20 роботов}
			\end{subfigure}
			&
			\begin{subfigure}{0.5\linewidth}
				\includegraphics[width = 1.0\columnwidth]{data/mean_paths/1000x1000/50.png}
			\caption*{50 роботов}
			\end{subfigure}
        \end{tabular}
        \caption*{Размер карты: 1000x1000}
	\end{table}

\end{document}
