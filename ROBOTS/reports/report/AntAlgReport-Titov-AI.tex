\documentclass{article}
\usepackage[utf8]{inputenc}
\usepackage[english,russian]{babel}
\usepackage{amsmath}
\usepackage{enumerate}
\usepackage[12pt]{extsizes}
\usepackage{xcolor,listings}
\usepackage[left=30mm, top=20mm, right=20mm, bottom=20mm, nohead, footskip=10mm]{geometry}


\usepackage[absolute,overlay]{textpos}
\usepackage{indentfirst}
\usepackage{float}
\restylefloat{table}
\usepackage{hyperref}
\usepackage{mathtext}
\usepackage{amsfonts}
\usepackage{amsthm}
\usepackage{tikz}
\usetikzlibrary{shapes,positioning,shadows,trees,automata,arrows.meta,shapes.geometric}
\usepackage{pgf-pie}
\usepackage{chngcntr}
\usepackage{pdfpages}
\usepackage{systeme}
\usepackage{empheq}
\counterwithin{figure}{section}

\pagestyle{plain}

\definecolor{String}{RGB}{134, 179, 0}
\definecolor{KeyColor}{RGB}{160,0,102}

\lstdefinestyle{style}{
	aboveskip=0pt,
	belowskip=10pt,
	showspaces=false,
	showstringspaces=false,
	basicstyle=\ttfamily\footnotesize,
	numbers=left,
	texcl=true,
	keywordstyle=\color{KeyColor},
	identifierstyle=\color{black},
	numberstyle=\scriptsize,
	stringstyle=\color{String},
	commentstyle=\color{gray},
	frame=tb
}

\begin{document}
    \thispagestyle{empty}
	\begin{center}
		Санкт-Петербургский политехнический университет Петра Великого\\
		Институт прикладной математики и механики\\
		Кафедра <<Телематика (при ЦНИИ РТК)>>\\
		\vspace*{\fill}
		\textbf{\Large{КУРСОВАЯ РАБОТА}}\\
		\vspace{0.5cm}
        \large{по дисциплине <<Семинар по роботизированным системам>>\\}
        \large{на тему <<Муравьиный алгоритм и алгоритм коллективного распределения целей>>}\\
        \vspace{1cm}
        по направлению 02.04.01.02 <<Организация и управление суперкомпьютерными системами>>
	\end{center}
	\vspace{3cm}
	\begin{tabular} {l l l}
	\hspace{10cm} & Выполнил: & Титов А.И.\\
	& Проверил: & Глазунов В.В.
	\end{tabular}
	\vspace*{\fill}
	\begin{center}
		Санкт-Петербург\\
		2019
    \end{center}
    \newpage


	\renewcommand\contentsname{Оглавление}
	\tableofcontents

	\newpage
	\addcontentsline{toc}{section}{ПОСТАНОВКА ЗАДАЧИ}
	\section*{ПОСТАНОВКА ЗАДАЧИ}

    В рамках курсовой работы требуется реализовать алгоритмы для построения оптимальных путей от роботов до целей. Для этого требуется реализовать следующие алгоритмы:

    \begin{enumerate}
        \item Алгоритм для построения реалистичной карты среды Diamond-Square;
        \item Алгоритм коллективного распределения целей
    \end{enumerate}

	\newpage
    \section{Описание алгоритмов}

    \subsection{Алгоритм Diamond-Square}

    \subsection{Муравьиный алгоритм}

    \subsection{Алгоритм коллективного распределения целей}

    \newpage
    \section{Программная реализация}

	\newpage
    \section{Результаты}

    \newpage
    \addcontentsline{toc}{section}{ЗАКЛЮЧЕНИЕ}
    \section*{ЗАКЛЮЧЕНИЕ}

    \newpage
	\renewcommand\refname{ЛИТЕРАТУРА}
	\addcontentsline{toc}{section}{ЛИТЕРАТУРА}
	\begin{thebibliography}{}
		\bibitem{Plan} Каляев И.А. Модели и алгоритмы коллективного управления в группах роботов // Физматлит, 2009, 279с.
	\end{thebibliography}


    \newpage
    \addcontentsline{toc}{section}{ПРИЛОЖЕНИЕ А. Исходный код}
    \section*{ПРИЛОЖЕНИЕ А. Исходный код}

    \newpage
    \addcontentsline{toc}{section}{ПРИЛОЖЕНИЕ Б. Таблицы замеров времени}
    \section*{ПРИЛОЖЕНИЕ Б. Таблицы замеров времени}

    \newpage
    \addcontentsline{toc}{section}{ПРИЛОЖЕНИЕ В. Графики решений}
    \section*{ПРИЛОЖЕНИЕ В. Графики решений}

\end{document}
