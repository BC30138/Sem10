\documentclass{article}
\usepackage[utf8]{inputenc}
\usepackage[english,russian]{babel}
\usepackage{amsmath}
\usepackage{enumerate}
\usepackage[12pt]{extsizes}
\usepackage{xcolor,listings}
\usepackage[left=30mm, top=20mm, right=20mm, bottom=20mm, nohead, footskip=10mm]{geometry}


\usepackage[absolute,overlay]{textpos}
\usepackage{indentfirst}
\usepackage{float}
\restylefloat{table}
\usepackage{hyperref}
\usepackage{mathtext}
\usepackage{amsfonts}
\usepackage{amsthm}
\usepackage{tikz}
\usetikzlibrary{shapes,positioning,shadows,trees,automata,arrows.meta,shapes.geometric}
\usepackage{pgf-pie}
\usepackage{chngcntr}
\usepackage{pdfpages}
\usepackage{systeme}
\usepackage{empheq}
\counterwithin{figure}{section}

\pagestyle{plain}

\definecolor{String}{RGB}{134, 179, 0}
\definecolor{KeyColor}{RGB}{160,0,102}

\lstdefinestyle{style}{
	aboveskip=0pt,
	belowskip=10pt,
	showspaces=false,
	showstringspaces=false,
	basicstyle=\ttfamily\footnotesize,
	numbers=left,
	texcl=true,
	keywordstyle=\color{KeyColor},
	identifierstyle=\color{black},
	numberstyle=\scriptsize,
	stringstyle=\color{String},
	commentstyle=\color{gray},
	frame=tb
}

\begin{document}
    \thispagestyle{empty}
	\begin{center}
		Санкт-Петербургский политехнический университет Петра Великого\\
		Институт прикладной математики и механики\\
		Кафедра <<Телематика (при ЦНИИ РТК)>>\\
		\vspace{7cm}
		\textbf{\Large{КУРСОВАЯ РАБОТА}}\\
		\vspace{0.5cm}
		\large{по дисциплине <<Проектирование приложений под ОС UNIX>>\\}
	\end{center}
	\vspace{3cm}
	\begin{tabular} {l l l}
	\hspace{10cm} & Выполнил: & Титов А.И.\\
	& Проверил: & Глазунов В.В.
	\end{tabular}
	\vspace{7cm}
	\begin{center}
		Санкт-Петербург\\
		2019
    \end{center}


	\renewcommand\contentsname{Оглавление}
	\tableofcontents

	\newpage
	\addcontentsline{toc}{section}{Постановка задачи}
	\section*{Постановка задачи}

	\begin{enumerate}
		\item Написать скрипт на языке \textbf{bash} для поиска файлов в указанной директории не новее установленной даты. Дата передается в формате <<дд/мм/гггг>>.
		\item Нужно написать демон под линукс который бы читал
		конфигурационный файл, брал оттуда параметры: каталог,
		время проверки. После чего проверял заданий
		каталог(рекурсивно), через заданные промежутки времени, на
		предмет наличия/отсутствия модификации файлов и
		записывал результаты в файл журнала, предусмотреть
		обработку сигналов: SIGHUP - для перечитывания
		конфигурационного файла, и SIGTERM - для контролируемого
		завершения демона (запись о выходе в файл журнала).
		\item Написать веб-сервер работающий в виде демона,
		веб-сервер должен обеспечивать базовую поддержку протокола
		HTTP и отдавать статический контент, обязательна реализация
		метода GET, по желанию методы HEAD и POST.
		Предусмотреть контроль и журналирование ошибок (либо в
		файл либо через syslog). Обеспечить одновременную работу
		сервера с множественным числом клиентов.
		\item Написать сетевой чат, сервер должен быть
		реализован в виде демона, предусмотреть контроль и
		журналирование ошибок, либо через syslog, либо в файл
		журнала. Сервер должен обеспечивать прием тестовых
		сообщений и дальнейшую пересылку их всем участникам.
		Реализовать сетевой клиент для проверки работоспособности
		сервера.
	\end{enumerate}

	\newpage
	\section{Bash-скрипт}

	\begin{lstlisting}[language = bash, style=style, title={Скрипт на языке bash}]
#!/bin/bash

error_func () {
    echo Error: Incorrect arguments. Type -h to get help.
    exit 1
}

usage="$(basename "$0")
    Something like smart search

    Usage:
		$(basename "$0") <directory> <date>
		search files in <directory> not newer then <date>.
                                Where <date> should have format dd/mm/yyyy.

    Options:
        -h --help       show this help text"

case $1 in
    -h|--help)
        if [ $# -ne 1 ]; then
            error_func
        fi
        echo "$usage"
        exit
        ;;
    *)
        if [ $# -ne 2 ]; then
            error_func
        fi

        if [ ! -d "$1" ]; then
            echo Error: Search directory should exist.
            exit 1
        fi
        ;;
esac

DIR="$( cd "$( dirname "${BASH_SOURCE[0]}" )" >/dev/null 2>&1 && pwd )"
IFS='/' read -ra datearray <<< "$2"
date=${datearray[2]}"-"${datearray[1]}"-"${datearray[0]}

mkdir tmp #question about if such directory exists, maybe remove this line?
tmp="tmp.$RANDOM"
touch --date $date ./tmp/$tmp

find $1 -type f -not -newer ./tmp/$tmp ! -iname $tmp

rm -rf tmp
	\end{lstlisting}

	\newpage
	\section{Демон-процесс}

	\begin{lstlisting} [language = C++, style=style, title = {Исходный код на языке C++}]
#include <stdio.h>
#include <stdlib.h>
#include <unistd.h>
#include <sys/stat.h>
#include <syslog.h>
#include <signal.h>
#include <dirent.h>
#include <string.h>
#include <time.h>
#include "../tools/tools.h"

char * dir_path;
int recheck_time;
FILE *config_file;
char *config_path;

void read_config(){
	char * strtime;
	size_t path_len = 0;
	size_t strtime_len = 0;

	config_file = fopen(config_path, "r");
	if (config_file == NULL) {
		syslog(LOG_ERR, "config file not found:
		path to config should be as argument");
		exit(EXIT_FAILURE);
	}

	if (getline(&dir_path, &path_len, config_file) == -1) {
		syslog(LOG_ERR, "config file formatting is incorrect");
		exit(EXIT_FAILURE);
	}

	if (getline(&strtime, &strtime_len, config_file) == -1) {
		syslog(LOG_ERR, "config file formatting is incorrect");
		exit(EXIT_FAILURE);
	}

	char *tmp;
	size_t len = 0;
	if (getline(&tmp, &len, config_file) != -1) {
		syslog(LOG_ERR, "config file formatting is incorrect");
		exit(EXIT_FAILURE);
	}

	recheck_time = atoi(strtime);
	strtok(dir_path, "\n");

	fclose(config_file);
}

void signal_handler(int sig) {
	switch (sig)
	{
		case SIGHUP:
			read_config();
			syslog(LOG_WARNING, "config file updating complete");
			break;
		case SIGTERM:
			syslog(LOG_WARNING, "daemon terminated");
			exit(0);
			break;
	}
}

void run_task_2(char *argv[]) {

	daemonize_();

	openlog("task2daemon", 0, LOG_USER);
	// работа с содержимым конфига
	DIR *dir;
	struct dirent *entry;
	struct stat file_stat;

	config_path = argv[1];
	read_config();
	signal(SIGTERM, signal_handler);
	signal(SIGHUP, signal_handler);
	while(1){
		dir = opendir(dir_path);
		while ((entry = readdir(dir)) != NULL) {
			if ((strcmp(entry->d_name, ".") != 0)
					&& (strcmp(entry->d_name, "..") != 0)) {
				char *file_path = malloc(strlen(dir_path)
						+ strlen(entry->d_name) + 2);
				strcpy(file_path, dir_path);
				strcat(file_path, "/");
				strcat(file_path, entry->d_name);
				stat(file_path,  &file_stat);
				if ((time(NULL) - file_stat.st_mtime)
					< recheck_time ) {
					syslog(LOG_NOTICE,
						"content of folder modified in
						 file: %s", entry->d_name);
				}
			}
		}
		closedir(dir);
		sleep(recheck_time);
	}
}
	\end{lstlisting}

	\section{Http-демон}

	\begin{lstlisting} [language = C++, style=style, title = {Исходный код на языке C++}]
#include <iostream>
#include <fstream>
#include <stdio.h>
#include <stdlib.h> // некоторые важные переменные, такие EXIT\_SUCCESS и переменные, вроде size\_t
#include <unistd.h> // POSIX стандартные типы
#include <sys/stat.h> // для того, чтобы варнинга не было о том, что не стоит
                      // использовать umask заданный неявно
#include <syslog.h> // для работы с журналом
#include <signal.h> // для работы с сигналами
#include <dirent.h> // для работы с файловой системой
#include <string.h>
#include "../tools/tools.h"
#include <sys/socket.h>
#include <netinet/in.h>



using namespace std;

string answerHeadHTML = "HTTP/1.1 200 OK\r\nVersion: HTTP/
1.1\r\nContent-Type: text/html; charset=utf-8\r\nContent-Length: ";
string answerHeadPicture = "HTTP/1.1 200 OK\r\nVersion: HTTP/
1.1\r\nContent-Type: image/png\r\nContent-Length: ";
string answerNotFound = "HTTP/1.1 404 Not Found\r\nConnection:
close\r\nVersion: HTTP/1.1\r\nContent-type: text/
html\r\nContent-length: ";

int listener;

void signal_handler_task_3(int sig) {
    switch (sig)
    {
    case SIGTERM:
        syslog(LOG_WARNING, "Daemon terminated!");
        close(listener);
        closelog();
        exit(0);
        break;
    }
}

void getData(string& tmpStr)
{
    tmpStr.erase(tmpStr.begin(), tmpStr.begin() + 5);
    int count = 0;
    while(tmpStr.at(count) !='\r')
        count++;
    tmpStr.erase(tmpStr.begin() + count, tmpStr.end());
    tmpStr.erase(tmpStr.end() - 9, tmpStr.end());
}

void run_task_3() {
    char path[1024];
    getcwd(path, sizeof(path));
    strcat(path, "/data");

    daemonize_(path);
    openlog("task3daemon", 0, LOG_DAEMON);
    syslog(LOG_WARNING, "daemon has been started");
    signal(SIGTERM, signal_handler_task_3);

    int sock;
    struct sockaddr_in addr;
    char buff[1024];// messages
    listener = socket(AF_INET, SOCK_STREAM, 0);

    if(listener < 0){
        syslog(LOG_WARNING, "Error while creating socket.");
        exit(1);
    }

    addr.sin_family = AF_INET;
    addr.sin_port = htons(8485);
    addr.sin_addr.s_addr = INADDR_ANY;

    if(bind(listener, (struct sockaddr *)&addr, sizeof(addr)) < 0){
        syslog(LOG_WARNING, "Socket is already used.");
        exit(2);
    }

    listen(listener, 1);
    while(1){
        sock = accept(listener, NULL, NULL);
        if(sock < 0){
            syslog(LOG_WARNING, "Socket can not be accepted.");
            exit(3);
        }
        switch(fork()){
            case -1:
                syslog(LOG_WARNING, "Fork can not be created.");
                break;
            case 0:
            {
                close(listener);
                if(recv(sock, buff , 1024, 0) < 0)
                    syslog(LOG_WARNING, "Data hasn't been recieved.");

                string recieveData = buff;//Полученный GET запрос
                string body;//Тело http ответа
                string answer;//http ответ

                getData(recieveData);//Путь запрашиваемого файла

                if(recieveData.find(".html") != std::string::npos)
                    answer = answerHeadHTML;
                else if(recieveData.find(".png") != std::string::npos)
                    answer = answerHeadPicture;


                std::ifstream webpage(recieveData);
                if (webpage.fail()){
                    answer = answerNotFound;
					body = "<HTML><HEAD><TITLE>404 - Not Found
					</TITLE></HEAD><BODY BGCOLOR=\"FFFFFF\">
					<H1>404 - Not Found</H1><HR>
					</BODY></HTML>";
                }
                else {
                    string text((
                                istreambuf_iterator<char>(webpage)),
                            (istreambuf_iterator<char>()));

                    body = text;
                }
                answer+=to_string(body.size());
                answer+="\r\n\r\n";
                answer+=body;
                if(send(sock, answer.c_str() ,answer.size(), 0) < 0)
                    syslog(LOG_WARNING, "Data hasn't been sent.");
                close(sock);
                exit(0);
            }
            default:
                close(sock);
        }
    }
}
	\end{lstlisting}

	\newpage
	\section{Демон-чат}
	\begin{lstlisting} [language = C++, style=style, title = {Исходный код на языке C++}]
#include<iostream>
#include<sys/types.h>
#include<sys/stat.h>
#include<sys/socket.h>
#include<netinet/in.h>
#include<stdio.h>
#include<unistd.h>
#include<fcntl.h>
#include<set>
#include <algorithm>
#include<syslog.h>
#include<signal.h>
#include<vector>

#include "../tools/tools.h"

using namespace std;

#define port 8087

int listener;
void signal_handler_server(int SIG) {
    switch (SIG)
    {
    case SIGTERM:
        syslog(LOG_WARNING, "Daemon has been terminated.");
        close(listener);
        closelog();
        exit(0);
    default:
        break;
    }
}

void launch_server() {
    char path[1024];
    getcwd(path, sizeof(path));
    strcat(path, "/data");
    daemonize_(path);
    vector<string> u_names;

    openlog("ChatServer", 0, LOG_WARNING);

    signal(SIGTERM, signal_handler_server);

    syslog(LOG_WARNING, "daemon has been started.");

    struct sockaddr_in addr;
    char buf[1024];
    int bytesRead;

    listener = socket(AF_INET, SOCK_STREAM, 0);
    if(listener < 0){
        syslog(LOG_WARNING, "Socket can't be created.");
        exit(1);
    }

    fcntl(listener, F_SETFL, O_NONBLOCK);
    addr.sin_family = AF_INET;
    addr.sin_port = htons(port);
    addr.sin_addr.s_addr = htonl(INADDR_LOOPBACK);

    if(bind(listener, (struct sockaddr*)&addr, sizeof(addr)) < 0){
        syslog(LOG_DAEMON, "Socket is already used.");
        exit(2);
    }

    listen(listener, 2);
    set<int> clients;

    while(1){
        //Заполняем множество сокетов
        fd_set readset;
        FD_ZERO(&readset);
        FD_SET(listener, &readset);

        for(set<int>:: iterator it = clients.begin(); it != clients.end(); it++)
            FD_SET(*it, &readset);
        //Задаем таймаут
        timeval timeout;
        timeout.tv_sec = 15;
        timeout.tv_usec = 0;

        //Ждем событие в одном из сокетов
        int mx = max(listener, *max_element(clients.begin(), clients.end()));
        if(select(mx+1, &readset, NULL, NULL, &timeout) < 0){
            syslog(LOG_DAEMON, "Select problem.");
            exit(3);
        }

        //Определяем тип события и выполняем соответствующие действия.
        if(FD_ISSET(listener, &readset)){
            //Поступил новый запрос на соединение, используем accept
            int sock = accept(listener, NULL, NULL);
            if(sock < 0){
                syslog(LOG_DAEMON, "Accept error.");
                exit(3);
            }
            fcntl(sock, F_SETFL, O_NONBLOCK);
            clients.insert(sock);
        }
        for(set<int>::iterator it = clients.begin(); it != clients.end(); it++){
            if(FD_ISSET(*it, &readset)){
                //Поступили данные от клиента, читаем их
                bytesRead = recv(*it, buf, 1024, 0);
                if(bytesRead <= 0){
                    //Соединения разорвано удаляем сокет из множества
                    close(*it);
                    clients.erase(*it);
                    continue;
                }

                string req = buf;
                string delimiter = "#";
                string u_req = req.substr(0, req.find(delimiter));
                if (u_req == "ureq") {
                    req.erase(0, 5);
			if (find(u_names.begin(), u_names.end(), req)
				!= u_names.end()){
                        string ex_name_err = "uname_error";
                        send(*it, ex_name_err.data(), ex_name_err.size(), 0);
                    }
                    else {
                        u_names.push_back(req);
                    }
                }
                else {
                    //Отправляем данные
			for(set<int>::iterator member = clients.begin();
					member != clients.end(); member++){
                        if(member != it)
                            send(*member, buf, bytesRead, 0);
                    }
                }
            }
        }
    }
}

string username;
int client_sock;
void *read (void *dummyPt)
{
    while(true){

        char buf[1024] = {'\0'};
        recv(client_sock, buf, sizeof(buf) + username.size(), 0);
        string recieveData = buf;

        if (recieveData.size() != 0) {
            if (recieveData == "uname_error") {
                cout << "\nBad name!\n";
                exit(3);
            }
            else {
                cout << "\n" << recieveData;
            }
        }
    }
}

int launch_client() {
    pthread_t threads[1];
    pthread_create(&threads[0], NULL, read, NULL);
    pthread_detach(threads[0]);

    struct sockaddr_in addr;
    client_sock = socket(AF_INET, SOCK_STREAM, 0);
    if(client_sock < 0){
        perror("Socket can't be created");
        exit(1);
    }

    addr.sin_family = AF_INET;
    addr.sin_port = htons(port);
    addr.sin_addr.s_addr = htonl(INADDR_LOOPBACK);

    if(connect(client_sock, (struct sockaddr*)&addr, sizeof(addr)) < 0){
        perror("Socket can't be connected");
        exit(2);
    }

    cout << "\033[1;36mEnter username: \033[0m";
    cin >> username;
    string req = "ureq#";
    req += username;
    send(client_sock, req.data(), req.size(), 0);


    while(1){
        string mess;
        cout << "\033[1;32m" << username << "#\033[0m";
        getline(cin, mess);
        mess+="\n";

        if(mess == string(":exit\n")){//Выход из чата
            close(client_sock);
            return 0;
        }
        else{
            string result = username;
            result+="# ";
            result+=mess;
            mess = result;
            send(client_sock, mess.data(), mess.size(), 0);
        }
    }
    close(client_sock);

    return 0;
}
	\end{lstlisting}
\end{document}
